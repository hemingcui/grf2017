\vspace{-.15in}\section{Research Background} \label{sec:related}\vspace{-.075in}

TBD.

\vspace{-.15in}\subsection{Paxos Consensus} 
\label{sec:background}\vspace{-.075in}

TBD.

\vspace{-.15in}\subsection{RDMA} 
\label{sec:background}\vspace{-.075in}

TBD.

\vspace{-.15in}\subsection{Datacenter Computing: Schedulers and VMs} 
\label{sec:background}\vspace{-.075in}

Schedulers: TBD.

VM: TBD.


\vspace{-.15in}\subsubsection{Motivation of Objectives in this Proposal} 
\label{sec:out-study}\vspace{-.075in}

TBD.

\vspace{-.15in}\subsection{Related Work by Others} 
\label{sec:others-work}\vspace{-.075in}

\para{Primary-backup in VM.} 

\para{Various Consensus Protocols.}

\para{HPC.} HPC already has a replication study. Most of them are in the MP 
mannor so not our focus, although the techniques developed in this proposal can 
be applied to HPC.

\vspace{-.15in}\subsection{Related Work by the PI} 
\label{sec:my-work}\vspace{-.075in}
% 
% First emphasis debugging experience on concurrency. Program analysis.
% Then mention security exploits found in Woodpecker.
% Then mention runtime systems.

The PI is an expert on program analysis algorithms and 
tools~\cite{wu:pldi12, woodpecker:asplos13, repframe:apsys15}, reliable and 
secure multithreading runtime systems~\cite{smt:cacm, cui:tern:osdi10, 
peregrine:sosp11, parrot:sosp13}, and fault-tolerant distributed 
systems~\cite{crane:sosp15}. The PI's works are published in 
premier conferences on systems software (OSDI 2010, SOSP 2011, SOSP 2013, and 
SOSP 2015) and programming languages (PLDI 2012 and ASPLOS 2013). In connection 
to this project, the PI has developed a precise program slicing algorithm (a 
key technique for \textbf{Objective 2}) and two reliable runtime systems (key 
techniques for \textbf{Objective 3}).

% In the last four years, the PI has been collaborating 
% with Columbia and CMU researchers to build three \smt systems, 
% \tern~\cite{cui:tern:osdi10}, \peregrine~\cite{peregrine:sosp11}, and 
% \parrot~\cite{parrot:sosp13}, with each addressing distinct research 
% challenges. Notably, \parrot, our latest system, is the first \smt runtime 
% system that is fast (12.7\% mean overhead for all evaluated programs) on a wide 
% range of 100+ popular multi-threaded programs. We have 
% put all \parrot's source code and raw evaluation results on 
% \url{http://github.com/columbia/smt-mc} for future research and industrial 
% deployments. Due to \parrot's high practicality, we plan to leverage it in this 
% proposed \xxx system.

% To show \smt's potential, we have applied these systems to greatly improving 
% software reliability and security, including improving precision and simplicity 
% of program analysis and verification~\cite{wu:pldi12}, making debugging 
% concurrency errors much easier~\cite{cui:tern:osdi10}, and improving coverage 
% of model checking~\cite{parrot:sosp13}. Our work have also been leveraged by 
% the community: some techniques in our \tern system~\cite{cui:tern:osdi10} has 
% been used by University of Washington Seatle researchers on computing a small 
% set of thread interleavings covering all inputs~\cite{ics:oopsla13}, and our 
% \parrot runtime system has been integrated with a CMU software model 
% checker~\cite{dbug:spin11}.
% 
% Our experience on building \smt runtime sytems can address the two 
% aforementioned major research challanges of the \smt and SMR integration. In 
% addition, our techniques on program analysis, verification, and model checking 
% can be deployed in \xxx replicas and enhance their reliability and security.


