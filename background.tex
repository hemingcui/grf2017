
\section{Research Background} \label{sec:related}

This section first introduces the background of concurrency bugs and 
their consequences on leading to concurrency attacks (\S\ref{sec:background}), 
and then presents others' related work (\S\ref{sec:others-work}), and the PI's 
related work (\S\ref{sec:my-work}).

\subsection{Concurrency Bugs and Attacks in Multithreaded Programs} 
\label{sec:background}

% P1: multithreading, hard to get right, plagued with concurrency bugs.
Driven by the prevalance of the multi-core hardware, multithreaded programs 
have become one main stream in real-world software. Unfortunately, despite 
dedades of efforts on developing new theretical primitives and building 
reliability tools, these programs are still notoriously difficult to get right. 
Multithreaded programs are plagued with various types of concurrency bugs, 
including data races, atomicity violations, deadlocks, and order violations. 
These bugs can easily lead to severe program behaviors such as crashes, hangs, 
and wrong outputs.



% P2: Can lead to various exploits. Mention our initial study, types 
% of software, 
% and types of concurrency exploits. Just like bugs in single threaded program 
% leader to exploits, concurrency bugs can lead to concurrency attacks.
% Mention HotPart, found XX exploits. We: found YY exploits. patterns.
% Two examples.
Worse, recent study~\cite{con:hotpar12} on a wide range of real-world 
multithreade programs has shown that attackers can leverage these concurrency 
bugs to construct \emph{concurrency attacks}. Attackers can construct specific 
inputs to trigger concurrency bugs with high probability (\eg, with only tens 
of repeated executions), corrupt critical global memory shared by threads in 
a program, leverage the corrupted values in memory to bypass security 
checks in programs' source code, and finally inject malicious code, gain 
privileges, or bypass authentication procedures.

Unfortunately, most traditional defense techniques are weakened or completely 
bypassed by concurrency attacks because these techniques are mainly designed to 
protect single-threaded programs. For example, authough taint-tracking tools 
have proven effective on tracking inputs' flow to sensitive locations by 
adding security tags to data, these tools are weakened in the multithreading 
context because a race on data can corrupt the tags and bypass tracking. 
\S\ref{sec:others-work} discusses why concurrency attacks can weaken various 
traditional defense techniques in more details. In short, compared to the 
traditional single-threaded programs, multithreaded programs are much more 
difficult to get right.


% P3: A key reason: a program can run into too many thread interleavings at 
% runtime.
% Hard to make sure all thread intleravings are correct. Mention our previous 
% work Parrot, exponentially many.
Our study shows that a key reason why multithreaded programs are so difficult 
to get right is that, at run time, these programs may run into 
exponentially many possible thread interleavings (or \emph{schedules}) 
depending on the permutation orders of inter-thread communication operations 
(\eg, mutex locks). It is extremely challenging to understand, test, analyze, 
or verify this huge amount of schedules and make sure they are free of 
concurrency bugs and exploits. Therefore, state-of-the-art neither has a 
rigourous model on describing concurrency attacks, nor lacks effective 
approaches to detect and defense such attacks.

This \xxx project takes a thorough approach to cope with concurrency attacks.
First, \xxx aims to develop a rigourous model to describe such attacks 
(\S\ref{sec:model}). Unlike traditional models to deal with sequential attacks, 
our model must incorporate vulnerable thread interleavings and their security 
consequences. Second, leveraging this model, it aims to build an effective 
detection approach for developers to find as many as concurrency attacks in the 
software testing phase (\S\ref{sec:detect}). Third, \xxx aims to build an 
efficient, transparent, and robust runtime system to defense concurrency attacks 
in the software deployment phase (\S\ref{sec:defense}).

% P4: Then, challenges come: a concurrency 
% attack model should incorporate thread interleavings. A detection methond 
% should also consider the consequence of thread interleavings. A runtim 
% technique must think of ways to replicate and reduce the amount thread 
% interleavings.
% P4: TBD.


\subsection{Related Work by Others} \label{sec:others-work}

\para{Reliability tools.} There are various prior work on concurrency error 
detection~\cite{yu:racetrack:sosp,savage:eraser,racerx:sosp03,lu:muvi:sosp,
avio:asplos06,conmem:asplos10}, diagnosis~\cite{racefuzzer:pldi08,
ctrigger:asplos09,atomfuzzer:fse08}, 
and correction~\cite{dimmunix:osdi08,gadara:osdi08,wu:loom:osdi10,cfix:osdi12}. 
These work mainly focus on the concurrency errors themselves, while \xxx's 
concurrency attack detection approach mainly focuses on the security 
consenquences of conrrency bugs, so our detection approach is complementary to 
these prior work. These prior work can aslo be deployed on backup 
replicas of \xxx's runtime infrastructure and help \xxx detect concurrency 
errors. Moreover, the asynchronous replication nature of \xxx's runtime 
infrastructure can also mitigate the performance overhead of these powerful 
analyses~\cite{repframe:apsys15}.

\para{TOCTTOU attacks.} TOCTTTOU (Time Of Check To Time Of Use) 
attacks have been well studied for 
decades~\cite{bishop:tocttou,toctou:fast08,toctou:fast05,toctou:usec03}. These 
attacks target mainly the file system interface, and leverage the non-atomicity 
on time-of-check to time-of-use of a file to escalate illegal file access. Our 
\xxx project targets general concurency attacks, which is much broader and 
severe than typical TOCTTTOU attacks for two main reasons. First, TOCTTTOU 
mainly causes illegal file access, while general concurency attacks can cause a 
much broader range of security attacks, ranging from gaining root privileges, 
injecting malicious code, to bypassing authentication checks. Second, the root 
cause of TOCTTTOU attacks is mainly caused by incorrect use of file system API, 
while concurrency attacks are mainly caused by concurrency bugs with memory 
accesses, which is much more difficult to track and prevent.

\para{Traditional security defense techniques.} TBD.
Mention shadow memory approaches.
Mention security libraries.
Mention stack overflow detection.
Say these are mainly for single-threaded programs.

\para{State machine replication (SMR).}  SMR has been studied by the literature 
for decades, and it is recognized by both industry and academia as a powerful 
fault-tolerance technique in clouds and distributed 
systems~\cite{lamportclock,smr:tutorial}. As a common practice, SMR uses 
\paxos~\cite{paxos,paxos:simple,paxos:complex} and its popular engineering 
approaches~\cite{paxos:live,paxos:practical} as the consensus protocol to 
ensure that all replicas see the same input request sequence. Since consensus 
protocols are the core of SMR, a variety of study improve different aspects of 
consensus protocols, including performance~\cite{epaxos:sosp13,paxos:fast} and 
understandability~\cite{raft:usenix14}. Although \xxx's current implementation 
takes a popular engineering approach~\cite{paxos:practical} for practicality, 
it can also leverage other consensus protocols and approaches.

\para{\dmt and \smt systems.}  In order to make multi-threading easier to 
understand, test, analyze, and replicate, researchers have built two types of 
reliable multi-threading systems: (1) stable multi-threading systems (or 
\smt)~\cite{grace:oopsla09, dthreads:sosp11, determinator:osdi10} 
that aim toreduce the number of possible thread interleavings for program all 
inputs, and(2) deterministic multi-threading systems (or 
\dmt)~\cite{dpj:oopsla09,dmp:asplos09,kendo:asplos09,coredet:asplos10,
dos:osdi10,ddos:asplos13, ics:oopsla13} that aim to reduce the number of 
possible thread interleavings on each program input. Typically, these systems 
use deterministic logical clocks 
instead of nondeterministic physical clocks to make sure inter-thread 
communications (\eg, \mutexlock and accesses to global variables) can only 
happen at some specific logical clocks. Therefore, given the same or similar 
inputs, these systems can enforce the same thread interleavings and eventually 
the same executions. These systems 
have shown to greatly improve software reliability, including coverage of 
testing inputs~\cite{ics:oopsla13} and speed of recording 
executions\cite{dos:osdi10} for debugging.


% \para{Symbolic execution.} The combination of symbolic and
% concrete executions has been a hot research topic. Researchers
% have built scalable and effective symbolic execution
% systems to detect errors [16–18, 20, 28–30, 49,
% 54], block malicious inputs [21], and preserve privacy in
% error reports [19]. Compared to these systems, TERN applies
% symbolic execution to a new domain: tracking input
% constraints to reuse schedules.
% 
% \para{Program analysis.} TBD. Program slicing [49] is a general 
% technique to prune irrelevant statements from a program or trace. Recently,
% systems researchers have leveraged or invented slicing techniques
% to block malicious input [18], synthesize executions for better error
% diagnosis [57], infer source code paths from log messages for postmortem
% analysis [56], and identify critical inter-thread reads that
% may lead to concurrency errors [59]. Our determinism-preserving
% slicing technique produces a correct trace slice for multithreaded
% programs and supports multiple ordered targets.


\subsection{Related Work by the PI} \label{sec:my-work}

First emphasis debugging experience on concurrency. Program analysis.
Then mention security exploits found in Woodpecker.
Then mention runtime systems.

A key requirement to making SMR practically support multi-threading is that all 
replicas must run the same thread interleavings so that the replica executions 
won't diverge. \smt~\cite{smt:cacm}, an advanced reliable multi-threading 
runtime technique invented by my collaborators and me, meets this requirement. 
This is because \smt can greatly reduce the number of possible thread 
interleavings in multi-threaded programs for all inputs with low performance 
overhead, making replications of multi-threaded programs almost as easy as 
single-threaded ones. In the last four years, the PI has been collaborating 
with Columbia and CMU researchers to build three \smt systems, 
\tern~\cite{cui:tern:osdi10}, \peregrine~\cite{peregrine:sosp11}, and 
\parrot~\cite{parrot:sosp13}, with each addressing distinct research 
challenges. Notably, \parrot, our latest system, is the first \smt runtime 
system that is fast (12.7\% mean overhead for all evaluated programs) on a wide 
range of 100+ popular multi-threaded programs. We have 
put all \parrot's source code and raw evaluation results on 
\url{http://github.com/columbia/smt-mc} for future research and industrial 
deployments. Due to \parrot's high practicality, we plan to leverage it in this 
proposed \xxx system.

To show \smt's potential, we have applied these systems to greatly improving 
software reliability and security, including improving precision and simplicity 
of program analysis and verification~\cite{wu:pldi12}, making debugging 
concurrency errors much easier~\cite{cui:tern:osdi10}, and improving coverage 
of model checking~\cite{parrot:sosp13}. Our work have also been leveraged by 
the community: some techniques in our \tern system~\cite{cui:tern:osdi10} has 
been used by University of Washington Seatle researchers on computing a small 
set of thread interleavings covering all inputs~\cite{ics:oopsla13}, and our 
\parrot runtime system has been integrated with a CMU software model 
checker~\cite{dbug:spin11}.

Our experience on building \smt runtime sytems can address the two 
aforementioned major research challanges of the \smt and SMR integration. In 
addition, our techniques on program analysis, verification, and model checking 
can be deployed in \xxx replicas and enhance their reliability and security.


