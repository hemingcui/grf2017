\vspace{-.15in}\section{Research Background} 
\label{sec:background}\vspace{-.075in}

TBD.

% This section presents the background of consensus (\S\ref{sec:consensus}) and 
% datacenter computing infrastructures (\S\ref{sec:datacenter}), motivation of 
% objectives (\S\ref{sec:motivation}), others' related work 
% (\S\ref{sec:others-work}), and the PI and co-I's related work 
% (\S\ref{sec:my-work}).

\vspace{-.15in}\subsection{Big-data Computing Frameworks} 
\label{sec:bigdata}\vspace{-.075in}

DISC frameworks (\eg Spark~\cite{nsdi12:spark},
DryadLINQ~\cite{osdi08:dryad}, MapReduce \cite{mapreduce}) are popular for 
computations
on tremendous
amounts of data, to finish tasks like data analysis and machine learning.
Computations are split across hosts and run in parallel, such that
computation resources are efficiently used.
% Therefore, computation powers are beyond the limit of one single machine, as a 
large-scale
% cluster of machines can cooperate to process one computation task.
Shuffles are frequent and sometimes are performance bottlenecks in DISC.
% in these systems.
% Spark~\cite{nsdi12:spark} has data shuffling during
% reduce stages, graph processing frameworks (\eg Pregal~\cite{sigmod10:pregel})
% uses messages for node updates, and Parameter Server~\cite{osdi14:pserver} has
% parameter updates.

% Many-to-many transformations (\eg \func{groupByKey}, \func{join}
% and \func{aggregateByKey}) are prevalent in DISC. Each many-to-many
% transformation takes many input records and generates many output records.
% Given a buggy output record, extant data provenance 
% system~\cite{icse16:bigdebug,vldb15:titian,vldb16:output} will
% report all input records going through this transformation, including many
% irrelevant input records that generate other output records.


To avoid excessive computation, many DISC systems adopt the lazy transformation 
approach.
~\cite{pig:vldb08,nsdi12:spark,osdi08:dryad}.
Spark uses lazy transformations (\eg \func{map}) for efficiency,
and calls to these transformations only create a new data structure called 
\func{RDD} with \func{lineage}.
The real transformations are only triggered when collecting operations (\eg 
\func{collect},
\func{count}) are called. These collecting operations trigger transformations 
along
lineages, where unnecessary computations are avoided. \xxx leverages the lazy 
transformation
feature in DISC to present its new \lazyp technique (\chref{subsec:referencep}).

\vspace{-.15in}\subsection{DFT and Diff Privacy}
\label{sec:dft}\vspace{-.075in}

% Information Flow Tracking is initially proposed for preventing sensitive 
% information
% leakage~\cite{dawn05:taint}.
% IFT attaches a tag to a variable (or object),
% and this tag will propagate throughout the computation.
% For example, a variable-level dataflow tracking will involve combinations of
% tags of two variables in each instruction, using an IOR operation.
% Different granularities of computation may incur different levels
% of computation overhead. Lower level (\eg byte-level) tracking will consume
% a lot of resources, as each byte of data in an IFT system has its own 
% tags~\cite{libdft:vee12}.
% 
% Multiple research has been focusing on efficiency and applications of IFT.
% Shadowreplica~\cite{shadowreplica:ccs13} proposed to make use of the multicore 
% resources while SHIFT~\cite{hardwardtaint:isca08}
% suggests accelerating dataflow tracking with hardware support.
% Several research~\cite{mit07:coverage,fse12:dtam} adopts IFT for providing 
% debugging primitives
% to improve software reliability.
% IFT has been applied to various areas, such as preventing sensitive information 
% (\eg GPS data and contacts)
% leakage in cellphone~\cite{taintdroid:osdi10, cleanos:osdi12}, providing secure
% cloud services~\cite{cloudfence:raid13} and server program 
% runtime~\cite{libdft:vee12}.
% To the best of
% our knowledge, no IFT system exists for big-data.

\vspace{-.15in}\subsection{Intel SGX}
\label{sec:sgx}\vspace{-.075in}

TBD.

% \vspace{-.15in}\subsection{Motivation of objectives} 
% \label{sec:motivation}\vspace{-.075in}



% \vspace{-.15in}% hack, for the gaia footnote.
\subsection{Related work by others} 
\label{sec:others-work}\vspace{-.075in}

TBD.

\vspace{-.15in}\subsection{Related work by the PI and co-I} 
\label{sec:my-work}\vspace{-.075in}
% 
% First emphasis debugging experience on concurrency. Program analysis.
% Then mention security exploits found in Woodpecker.
% Then mention runtime systems.

The PI is an expert on reliable concurrent and distributed 
systems~\cite{smt:cacm, cui:tern:osdi10, peregrine:sosp11,
parrot:sosp13, crane:sosp15, tripod:apsys16}. The 
PI's works are published in premier conferences on systems (OSDI, SOSP, SOCC, 
TPDS, and ACSAC) and programming languages (PLDI and ASPLOS). The co-I is an 
expert on high-performance 
computing~\cite{powerrock,hwang,jessica,cheung,khokhar}, fault-tolerance~\cite{ 
sheng,shengdi1}, and VMs~\cite{rhymes,shengdi,jessica2}. The 
co-I's works are published in top systems conferences (Cluster '02, SC '13, 
and ICPADS '14) and journals (JPDC '00, TPDS '13, IEEE Tran. Computers '14). As 
preliminary results for this \xxx proposal, the PI has 
developed Kakute~\cite{kakute:acsac17} and TPDS (parts of \textbf{Objective 
1}). 


