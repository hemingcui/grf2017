\vspace{-.15in}\section{Research Background} 
\label{sec:background}\vspace{-.075in}

This section presents the background of consensus (\S\ref{sec:consensus}) and 
datacenter computing infrastructures (\S\ref{sec:datacenter}), motivation of 
objectives (\S\ref{sec:motivation}), others' related work 
(\S\ref{sec:others-work}), and the PI and co-I's related work 
(\S\ref{sec:my-work}).

\vspace{-.15in}\subsection{\paxos: tolerating single point of failures} 
\label{sec:consensus}\vspace{-.075in}

Distributed consensus (typically, 
\paxos~\cite{paxos,paxos:practical,paxos,paxos:simple,paxos:complex}) plays a 
core role in distributed systems, 
including fault-tolerance~\cite{eve:osdi12,rex:eurosys14,crane:sosp15},
leader election~\cite{zookeeper, chubby:osdi}, and ordering 
services~\cite{ellis:thesis,manos:hotdep10,scatter:sosp11}. A \paxos protocol
replicates the same application on a group of computers (or \emph{replicas}) 
and enforces same inputs for this application, as long as a quorum (typically, 
majority) of replicas work normally. If leader fails, it elects a new 
leader~\cite{paxos:practical}. Therefore, \paxos tolerates various faults (\eg, 
minor replica failures and packet losses). 

% \paxos is widely used in many systems. For instance, 
% Scatter~\cite{scatter:sosp11} runs 8$\sim$12 replicas in a
% \paxos group to order client requests, and it lets replicas reply requests
% in parallel. A bigger group size will improve Scatter throughput. 
% Moreover, state machine replication (SMR) 
% systems~\cite{eve:osdi12,rex:eurosys14,crane:sosp15} use \paxos to 
% improve the availability of server applications (\eg, \mysql~\cite{mysql}).

Unfortunately, an open challenge is that the consensus latency of existing 
\paxos protocols is too high, because their consensus messages go through OS 
kernels and software TCP/IP layers (a \vv{ping} round-trip takes about 200 \us 
in a 10Gbps network). \S\ref{sec:latency-problem} presents this problem in 
detail.

Recently, as Remote Direct Memory Access (RDMA) networks (\eg, 
Infiniband~\cite{infiniband}) become popular within a datacenter, 
RDMA appears a promising solution to address the \paxos performance challenge. 
The fastest type of RDMA operations called ``one-sided RDMA writes" (for short, 
\emph{WRITE} in the rest of this proposal) takes only about 3 
\us~\cite{pilaf:usenix14} in a round-trip, because it allows a local computer to 
directly writes to a remote computer's memory without involving OS kernel or CPU 
on the remote computer. This ultra low latency not only comes from its kernel 
bypassing feature, but also its dedicated network stack implemented in hardware. 
Therefore, RDMA is considered the fastest kernel bypassing 
technique~\cite{herd:sigcomm14,pilaf:usenix14,dare:hpdc15}; it is several times 
faster than software-only kernel bypassing techniques (\eg, 
DPDK~\cite{dpdk} and Arrakis~\cite{arrakis:osdi14}).

\vspace{-.15in}\subsection{Datacenter computing infrastructures}
\label{sec:datacenter}\vspace{-.075in}

More and more applications are run in a datacenter by two 
independent types of infrastructures. The first type is 
schedulers~\cite{borg:eurosys15,mesos:nsdi11,tupperware, yarn:socc13, 
autopilot:sosp07,quincy:sosp09,apollo:osdi14,fuxi:vldb14}. Although existing 
schedulers themselves have been made available via \paxos 
(\eg,~\cite{mesos:nsdi11}), their applications are not. Therefore, if failures 
such as computer hardware errors occur, these schedulers have to re-launch 
applications, leading to a substantial application downtime and thus a 
huge lost for mission-critical applications (\eg, financial platforms and social 
networks).

The second infrastructure type is VM~\cite{amazon:vpc, openstack, esx:osdi02, 
kvm, xen:sosp}. VM abstracts away the heterogeneous physical computing 
resources, making computing resources easy to utilize and balance loads (\eg, 
via live migration~\cite{vmotion:atc05,xen:migration:nsdi05}). VM is also known 
for its secure isolation on resources~\cite{xen:sosp,kvm,vmware:sugerman}. 
Although some VM fault-tolerance approaches such 
as primary-backup~\cite{remus:nsdi08,ftvm} exist, they consume much 
time and network bandwidth as they need to transfer all memory modified by 
applications across VM replications.
% Nowadays a computer is deploying more and more VMs due to its increasing CPU 
% cores, a computer hardware error can turn down all VMs and applications running 
% on them. 

These two infrastructures usually work independently depending on 
runtime trade-off such as performance or security isolation. Some applications 
are run solely by schedulers (e.g., Mesos~\cite{mesos:nsdi11} usually uses 
lightweight Linux containers, not VMs), and some other applications are run 
solely by VMs (e.g., Amazon EC2). Therefore, this proposal strengthens the 
two infrastructures respectively. If people need to run an application with 
both infrastructures, they can choose either infrastructure developed from 
this proposal and the other one from outside. For instance, people can choose 
our \tripod scheduler (\S\ref{sec:scheduler-arch}) and VMWare~\cite{esx:osdi02}.

% Many applications are mission-critical (\eg, financial platforms, social 
% networks, and medical services,), so they naturally desire both high 
% performance and availability. For instance, a trading platform must be highly 
% available during operation hours~\cite{nosql:finance}. Another instance is 
% social networks: minor computer failures in a Facebook datacenter led to 
% several outages in recent years~\cite{facebook:outage}.




% 


% Schedulers and VMs are largely complementary infrastructures depending on 
% application requirements such as performance and security. Many schedulers 
% use lightweight containers~\cite{docker,lxc} for isolation, while many other 
% cloud deployments (\eg, Amazon EC2) simply use VMs without schedulers. 
% Therefore, this proposal strengthens the two infrastructures respectively.


\vspace{-.15in}\subsection{Motivation of objectives} 
\label{sec:motivation}\vspace{-.075in}

% Several open problems. First, application is not highly-available. Second, when 
% application runs on large number of machines, existing replication systems, 
% which mainly target single-machine deployments, can not deploy such 
% applications with both properly replication and resource utilization 
% considerations. Therefore, we take a holistic approach: first create a fast, 
% scable, yet still general consensus protocol (Objective 1), and then we 
% integrate this protocol with two major infrastrutures (Objective 2 and 3). This 
% novel integratation provides a clean high-availability support for general  
% applications.
The objectives of \xxx\footnote{Gaia is an ancient Greek goddess who takes good 
care of everything on earth, including software applications.} stem from two 
research problems in datacenter computing. First, despite the core role and 
good deployments of \paxos, it suffers from high consensus latency. Therefore, 
\textbf{Objective 1} addresses this problem by leveraging RDMA to create a 
fast consensus algorithm and implementation protocol. Second, although many 
mission-critical applications demand stringent availability, existing 
infrastructures lack such support. To benefit these applications, 
\textbf{Object 2 and 3} take a holistic methodology to integrate our protocol 
with two major infrastructures, potentially benefiting almost all applications.


% \vspace{-.15in}% hack, for the gaia footnote.
\subsection{Related work by others} 
\label{sec:others-work}\vspace{-.075in}

\para{Various consensus protocols.}  
There are a rich set of \paxos 
algorithms~\cite{paxos:practical,paxos,paxos:simple,paxos:complex,
epaxos:sosp13} 
and implementations~\cite{paxos:live,paxos:practical,chubby:osdi,crane:sosp15}. 
Since consensus protocols play a core role in 
datacenters~\cite{matei:hotcloud11, mesos:nsdi11, 
datacenter:os} and worldwide 
distributed systems~\cite{spanner:osdi12,mencius:osdi08}, various works 
have been conducted to improve specific aspects of consensus protocols, 
including commutativity~\cite{epaxos:sosp13}, 
understandability~\cite{raft:usenix14,paxos}, and verifiable 
rules~\cite{modist:nsdi09,demeter:sosp11}. \paxos is notoriously difficult to be 
fast and scalable to a large consensus 
group~\cite{ellis:thesis,manos:hotdep10,scatter:sosp11}. 

To make \paxos's throughput scalable (\ie, more replicas, higher throughput), 
various systems leverage \paxos as a core building block to develop advanced 
replication approaches, including partitioning program 
states~\cite{scatter:sosp11,ssmr:dsn14}, splitting consensus 
leadership~\cite{mencius:osdi08,spaxos:srds12}, and hierarchical 
replication~\cite{manos:hotdep10,scatter:sosp11}. Theses approaches have 
improved throughput. However, the core of these systems, 
\paxos, still faces an unscalable consensus 
latency~\cite{ellis:thesis,scatter:sosp11,manos:hotdep10}. By realizing 
\textbf{Objective 1} (\S\ref{sec:protocol}), these systems can scale even 
better.

\para{Fault-tolerance in schedulers.} Datacenter
schedulers~\cite{borg:eurosys15,mesos:nsdi11,tupperware,yarn:socc13,
autopilot:sosp07,quincy:sosp09,apollo:osdi14,fuxi:vldb14} support diverse 
applications (\eg, Hadoop~\cite{hadoop}, Dryad~\cite{dryad}, and key-value 
stores~\cite{redis}). Existing schedulers mainly focus on obtaining high 
availability for themselves by replicating their own essential components, or 
focus on application recovery~\cite{fuxi:vldb14} instead of availability. 
To the best of our knowledge, no existing schedulers provide a
high availability support to general applications. \textbf{Objective 2} 
(\S\ref{sec:scheduler}) aims to provide this support.
% \xxx's consensus protocol design is general to application and 
% schedulers so it can be integrated in other systems.

%  Migration and primary-backup.
\para{Fault-tolerance in VM.} Two 
techniques, primary-backup~\cite{remus:nsdi08,ftvm} and live 
migration~\cite{vmotion:atc05,xen:migration:nsdi05}, exist for improving 
application reliability in VM. Primary-backup (\eg, vSphere~\cite{ftvm})
uses the hypervisor on a primary VM to track memory pages modified by 
applications, it then transfers these pages to a backup VM. Live migration 
uses a similar technique as primary-backup. These techniques can be used to 
improve computer load balance and consolidating many VMs on fewer computers to 
save datacenter energy. A common problem in the two techniques is that they 
incur substantial application downtime (\eg, 8 seconds in 
vMotion~\cite{vmotion:atc05}) and network bandwidth during the memory transfer. 
\textbf{Objective 3} (\S\ref{sec:vm}) will greatly mitigate this problem.

% This approach consumes much network bandwidth and has weaker 
% primary-consistency (\eg, if primary rejoins the network) than \paxos.

\para{RDMA techniques.} Recently RDMA is deployed in various types of 
datacenter networks, including Infiniband~\cite{infiniband} and 
RoCE~\cite{roce}. RDMA has been leveraged in many 
software systems to improve different performance aspects, including high 
performance computing~\cite{openmpi}, key-value 
stores~\cite{pilaf:usenix14,herd:sigcomm14,farm:nsdi14,memcached:rdma}, 
distributed transactions~\cite{drtm:sosp15,farm:sosp15}, and file 
systems~\cite{gibson:nfs}. These systems are complementary to \xxx objectives.

\vspace{-.15in}\subsection{Related work by the PI and co-I} 
\label{sec:my-work}\vspace{-.075in}
% 
% First emphasis debugging experience on concurrency. Program analysis.
% Then mention security exploits found in Woodpecker.
% Then mention runtime systems.

The PI is an expert on reliable concurrent and distributed 
systems~\cite{smt:cacm, cui:tern:osdi10, peregrine:sosp11,
parrot:sosp13, crane:sosp15, tripod:apsys16}. The 
PI's works are published in premier conferences on systems (OSDI '10, 
SOSP '11, SOSP '13, and SOSP '15) and programming languages (PLDI '12 and 
ASPLOS '13). As preliminary results for this \xxx proposal, the PI has 
developed a fast consensus protocol~\cite{crane:sosp15} (part of 
\textbf{Objective 1}) and a fault-tolerant prototype 
scheduler~\cite{tripod:apsys16} (part of \textbf{Objective 2}). The co-I is an 
expert on high-performance 
computing~\cite{powerrock,hwang,jessica,cheung,khokhar}, fault-tolerance~\cite{ 
sheng,shengdi1}, and VMs~\cite{rhymes,shengdi,jessica2}. The 
co-I's works are published in top systems conferences (Cluster '02, SC '13, 
and ICPADS '14) and journals (JPDC '00, TPDS '13, IEEE Tran. Computers '14).


