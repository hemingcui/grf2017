\vspace{-.15in}\section{Research Plan and Methodology}
\label{sec:rep}\vspace{-.075in}

This \xxx project strenghens the reliability of datacenter computing with a 
holistic, thorough methodology. To this end, this section first proposes a 
fast, scalable consensus protocol (\S\ref{sec:protocol}). With this protocol, 
it presents a scheduler that makes general applications fault-tolerant 
(\S\ref{sec:scheduler}) and a new VM replication architecture (\S\ref{sec:vm}). 
The first two objectives include preliminary results. Finally, this section 
describes our research plan (\S\ref{sec:plan}).

\vspace{-.15in}\subsection{Objective 1: Building Fast, Scalable Consensus via
RDMA} 
\label{sec:model}\vspace{-.075in}

% P1: as mentioned in background, a key reason is thread interleavings, 
% so we need to reason about the general patterns we have. Or we say our 
% methodology is just like pattern matching.
Traditional \paxos protocols incur high consensus latency because they run on 
TCP/IP, which go through OS kernels and software network layers. In this 
section, \S\ref{sec:problem} analyzes this latency problem and its poor 
scalability in detail, and then presents our new RDMA-based \paxos 
protocol called \falcon (\S\ref{sec:falcon}).

\vspace{-.15in}\subsubsection{Problem: Consensus Latency of existing \paxos 
protocols scale poorly} 
\label{sec:examples}\vspace{-.075in}

First, mainly introduce the problems in traditional protocols.

Second, briefly mention the problem in DARE. and its scalability bottleneck.

\vspace{-.15in}\subsubsection{Falcon: a fast, scalable RDMA-based \paxos 
protocol} 
\label{sec:falcon}\vspace{-.075in}

TBD.

Three figures. First, falcon arch. Second, consensus protocol. Third, falcon 
results compared to traditional ones and DARE.

Primilinary results: Crane. Falcon. Say Crane is first version. Falcon totally 
subsumes Crane. Falcon also has initial results.

\vspace{-.15in}\subsection{Objective 2: 
Integrating Falcon with Datacenter Schedulers}\label{sec:detect}\vspace{-.075in}

\vspace{-.15in}\subsubsection{\tripod: the fault-tolerant scheduler 
architecture} 
\label{sec:scheduler-arch}\vspace{-.075in}

TBD.


\vspace{-.15in}\subsubsection{The replication-aware resource allocation scheme}
\label{sec:detect-arch}\vspace{-.075in}

TBD.

Two figures: one is TRIPOD arch. The other is TRIPOD results from the workshop 
paper.

\para{Preliminary results.} Workshop paper.

\para{Future work.} New algorithm on scheduling and replication. Others?

\vspace{-.15in}\subsection{Objective 3: 
Strenghening VM to improve application 
availability}\label{sec:defense}\vspace{-.075in}


% TBD: need a new replication approach name.
\vspace{-.15in}\subsubsection{Idea I: Hybrid Replication} 
\label{sec:defense-arch}\vspace{-.075in}

TBD.

\vspace{-.15in}\subsubsection{Idea II: \paxos-based Live Migration} 
\label{sec:defense-arch}\vspace{-.075in}

TBD.

\vspace{-.15in}\subsection{Research Plan} \label{sec:plan}\vspace{-.075in}

This project will require two PhD students S1 and S2 to work for 
three years. In the first year, S1 will develop and refine the concurrency 
attack model (part of \textbf{Objective~1}), and S2 will leverage the model to 
design the detailed workflow of the detection approach (part of 
\textbf{Objective~2}) by working closely with S1. In the second year, S1 will 
do an empirical study on how well the model represents real-world concurrency 
attacks (part of \textbf{Objective~1}), and S2 will implement the detection 
approach as a software tool (part of \textbf{Objective~2}). In the third year, 
S1 will implement the defense infrastructure (\textbf{Objective~3}), and S2 
will study the detection tool on a broad range of real-world multithreaded 
programs to find new attacks.
% Both students will 
% involve theoretical methods, implement real software systems, and 
% perform real-world study.
% The PI will supervise the students by providing 
% advice concerning both theoretical and systems implementation levels.


