\vspace{-.15in}\section{Research Plan and Methodology}
\label{sec:rep}\vspace{-.075in}

This \xxx project tackles concurrency attacks with a thorough, systematic 
methodology. To this end, this section presents three objectives, including a 
general, rigorous concurrency attack model (\S\ref{sec:model}), a systematic 
concurrency attack detection approach (\S\ref{sec:detect}) by following this 
model, and a runtime defense infrastructure (\S\ref{sec:defense}). Each 
objective includes our preliminary results. Finally, this section describes our 
research plan (\S\ref{sec:plan}).

\vspace{-.15in}\subsection{Objective 1: Building Fast, Scalable Consensus via
RDMA} 
\label{sec:model}\vspace{-.075in}

% P1: as mentioned in background, a key reason is thread interleavings, 
% so we need to reason about the general patterns we have. Or we say our 
% methodology is just like pattern matching.
State-of-the-art lacks a general, rigorous model to describe how concurrency 
attacks manifest. In this section, \S\ref{sec:examples} gives two concurrency 
attack examples in our preliminary study (\S\ref{sec:our-study}), and then 
\S\ref{sec:attack-phase} proposes our model based on this study.

\vspace{-.15in}\subsubsection{Problem: Consensus Latency Scales poorly to 
Replica Group Size} 
\label{sec:examples}\vspace{-.075in}



\vspace{-.15in}\subsubsection{Falcon: A new RDMA-based Consensus Protocol} 
\label{sec:attack-phase}\vspace{-.075in}

TBD.

Three figures. First, falcon arch. Second, consensus protocol. Third, falcon 
results compared to traditional ones and DARE.

Primilinary results: Crane. Falcon. Say Crane is first version. Falcon totally 
subsumes Crane. Falcon also has initial results.

\vspace{-.15in}\subsection{Objective 2: 
Integrating Falcon with Datacenter Schedulers}\label{sec:detect}\vspace{-.075in}



TBD.


\vspace{-.15in}\subsubsection{TRIPOD Architecture}
\label{sec:detect-arch}\vspace{-.075in}

Two figures: one is TRIPOD arch. The other is TRIPOD results from the workshop 
paper.

\para{Preliminary results.} Workshop paper.

\para{Future work.} New algorithm on scheduling and replication. Others?

\vspace{-.15in}\subsection{Objective 3: 
Strenghening VM to improve application 
availability}\label{sec:defense}\vspace{-.075in}


\vspace{-.15in}\subsubsection{Idea I: Hybrid Replication} 
\label{sec:defense-arch}\vspace{-.075in}

TBD.

\vspace{-.15in}\subsubsection{Idea II: Paxos-based Live Migration} 
\label{sec:defense-arch}\vspace{-.075in}

TBD.

\vspace{-.15in}\subsection{Research Plan} \label{sec:plan}\vspace{-.075in}

This project will require two PhD students S1 and S2 to work for 
three years. In the first year, S1 will develop and refine the concurrency 
attack model (part of \textbf{Objective~1}), and S2 will leverage the model to 
design the detailed workflow of the detection approach (part of 
\textbf{Objective~2}) by working closely with S1. In the second year, S1 will 
do an empirical study on how well the model represents real-world concurrency 
attacks (part of \textbf{Objective~1}), and S2 will implement the detection 
approach as a software tool (part of \textbf{Objective~2}). In the third year, 
S1 will implement the defense infrastructure (\textbf{Objective~3}), and S2 
will study the detection tool on a broad range of real-world multithreaded 
programs to find new attacks.
% Both students will 
% involve theoretical methods, implement real software systems, and 
% perform real-world study.
% The PI will supervise the students by providing 
% advice concerning both theoretical and systems implementation levels.


