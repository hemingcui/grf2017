\para{GAIA: Strenghtening the Reliability of Datacenter Computing with 
Fast and Scalable Distributed Consensus}

% \newline
\para{Abstract:}

% P1: More and more pplications are being run in datacenters with schedulers or 
% virtual machines. Many applications are mission critical. However, existing 
% schedulers or virtual machines did not provide satisfactory fault tolerence 
% to applications, causing big unavailable time window.
To deal with the rapidly increasing volume of data, more and more software
applications run within a datacenter containing numerous computers. To 
harness the massive datacenter computing resources, applications are deployed 
with two major types of infrastructures: schedulers and virtual machines. 
These infrastructures have brought many benefits, including improving resource 
utlization, balancing loads, and saving energy. Unfortunately, as an application 
runs on more computers, minor computer failures will occur more likely and can 
turn down the entire application, causing severe disasters such as the 2015 
NYSE trading halts and recent Facebook outages. Existing infrastructures lack 
support to ensure high-availability for applications.

% P2: This project aims to greatly improve the availability of datacenter 
% computing % with three objectives with a computer theory called state machine 
% replication. Introduce basic of the theory: replica, majority, consensus, 
% etc. 
This GAIA project takes a holistic approach to greatly improve application 
availability with three objectives. First, we will create a fast, scalable 
distributed consensus protocol for general applications. Distributed consensus 
consensus provides strong fault-tolerance: it replicates the same application 
on different computers and always enforces same inputs on these computers, as 
long as a majority of computers are alive. An open challenge is that 
traditional consensus protocols are too slow (over 300 micro-seconds) because 
their messages go through OS kernels and software TCP/IP layers. We tackle this 
challenge by creating a fast consensus protocol with an advanced networking 
technique called Remote Direct Memory Access (RDMA). Our preliminary results 
published in [SOSP '15] show that our protocol can support diverse, 
unmodified applications, and our latest protocol was 20X~31X faster than 
traditional protocols even running on 35X more computers.


% P3: First, build a fast, scalable SMR protocol. 
% Key challenge, scalability. We propose a new RDMA-based % protocol. Making 
% replicas agree on inputs on bare memory. Also, our protocol is general, cite 
% Crane.

% P4: Second, strength schedulers with this protocol. Challenge: ? Our approach?
Second, we will construct the first fault-tolerent scheduler by integrating our 
protocol with popular datacenter schedulers. In order to seamleesly achieve 
resource allocation and application replication, we propose a new 
replication-aware resource allocation workflow for the scheduler. Preliminary 
results published in [APSys '16] show that our scheduler can efficiently 
support Redis, a widely used key-value store.

Third, we will make our protocol and virtual machines (VM) form a 
mutual-benefitual eco-system. This eco-system not only leverages the VM 
hypervisor layer to automatically replicate applications, but it introduces a 
new VM live migration approach for computer load balance. To migrate 
application execution states to remote computers, prior live migration 
approachs incur substantial application down time and resource consumption on 
local computer. With our new approach, we need only migrate a consensus 
leadership, which consumes almost zero time and resource.


% P6: We envision that... impact, of the project.
We envision that our expertise on building reliable distributed systems and 
our preliminary results will help us achieve all the three objectives. By 
greatly improving the availability of many datacenter applications, this project 
will benefit almost all computer users and software vendors, including social 
networking and financial platforms. This project will also advance various 
datacenter techniques (\eg, migration) and attract researchers to build more 
fault-tolerant infrastructures.


