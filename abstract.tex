\para{GAIA: Greatly Improving the Reliability of Datacenter Computing with 
Fast, Scalable Distributed Consensus}

% \newline
\para{Abstract:}

% P1: More and more pplications are being run in datacenters with schedulers or 
% virtual machines. Many applications are mission critical. However, existing 
% schedulers or virtual machines did not provide satisfactory fault tolerence 
% to applications, causing big unavailable time window.
To deal with the rapidly increasing volume of data, more and more software
applications run within a datacenter with numerous computers. To harness the 
massive datacenter computing resources, applications are deployed with two major 
types of infrastructures: schedulers and virtual machines. These 
infrastructures have brought many benefits, including improving resource 
utlization, balancing loads, and saving energy. Unfortunately, as an application 
runs on more computers, minor computer failures will happen more likely and can 
turn down the entire application, causing severe disasters such as the NYSE 
Trading halts in 2015 and several Facebook outages in recent years. Existing 
infrastructures lack support to ensure high-availability for applications.

% P2: This project aims to greatly improve the availability of datacenter 
% computing % with three objectives with a computer theory called state machine 
% replication. Introduce basic of the theory: replica, majority, consensus, 
% etc. 
This GAIA project takes a holistic approach to improve the the availability of 
datacenter applications, with three objectives. First, we will create a fast, 
scalable distributed consensus protocol for general applications. Distributed 
consensus consensus is a fault-tolerance theory: it replicates the same 
application on different computers and always enforces the same input for the 
program, as long as a majority of computers are alive and agree on the input. A 
key challenge is that traditional consensus protocols are too slow (over 300 
micro seconds), because their messages go through OS kernels and software 
TCP/IP layers. We tackle this challenge by creating a fast consensus protocol 
with an advanced networking technique called Remote Direct Memory Access 
(RDMA). Our initial results published in [SOSP '15] showed that our protocol 
can support diverse real-world applications, and our latest protocol was 
20X~31X faster than traditional protocols even running on 35X more computers.


% P3: First, build a fast, scalable SMR protocol. 
% Key challenge, scalability. We propose a new RDMA-based % protocol. Making 
% replicas agree on inputs on bare memory. Also, our protocol is general, cite 
% Crane.

% P4: Second, strength schedulers with this protocol. Challenge: ? Our approach?
Second, we will construct the first fault-tolerent scheduler by integrating our 
new protocol with widely used datacenter schedulers. In order to seamleessly 
achieve computation replication by our protocol and resource allocation by the 
schedule, we propose a replication-aware, resource allocation workflow for the 
scheduler. Our initial results published in [APSys '16] showed that this 
scheduler can efficiently support Redis, a key-value store application widely 
used in social network and financial trading platforms.

% Eco-system.
% P5: Third, strenghten virtual machine with this protocol. Challenge: it is 
% difficult to enforce execute states efficiently at the same 
% time. We: hybrid replication. Or some other name. Fine-grained replication?
% Third, we will make our protocol and virtual machines form a novel 
% mutual-beneficial eco-system. This eco-system replicates the same virtual 
% machine on different computers and enforces same external inputs 
% across these virtual machine copies. With benefits from the hypervior layer, 
% our protocol can efficiently, precisely detect and resolve different 
% application execution states across virtual machines and make them run in sync. 
% With the help from our protocol, virtual machine migration, a widely deployed 
% load balance scheme, can enjoy much shorter operation time and much fewer 
% transfered states during migration.

Third, we will make our protocol and VMs form a novel eco-system. This 
eco-system not only leverages the hypervisor layer to automatically replicates 
applications running in VMs, but it leads to a new VM live migration approach 
which is designed for load balance. In order to migrate an application's 
execution states to remote computers, prior live migration approachs incur 
substantial application down time and resource consumption on local computer. In 
our new approach, we only need to migrate the leadership of our protocol, which 
consumes almost zero time and resource.


% P6: We envision that... impact, of the project.
We envision that our expertise on building reliable distributed protocols and 
our initial results will help us achieve all the three objectives. By greatly 
improving the availability of many applications running in a datacenter, this 
project will benefit almost all computer users and software vendors, including 
social networking, military software services, and financial platforms. This 
project will also greatly advance existing datacenter techniques (\eg, 
migration) and attract more researchers to build more fault-tolerant 
infrastructures.


